% OLD MATH PROOF SECTION
Considering the sign and step function properties given in \req{NFF2}, we see that \req{NFF1} is equal to $1/2$ when $x=0$. This is the expected behavior of the FD distribution when the energy is equal to the chemical potential. To further demonstrate that \req{NFF1} is indeed the FD distribution, we will use the following properties of the sign and step functions
\begin{align}
\label{NFF2a}
\sgn(x)\equiv\frac{|x|}{x}\equiv\frac{x}{|x|}\,,\quad
\sgn(x)=2\Theta(x)-1\,,\quad
\Theta(x)+\Theta(-x)=1\,.
\end{align}
We also write the following hyperbolic expression
\begin{equation}
\label{NFFa1}
\sgn^{2}(x)\sinh(x)=\sinh(x)\,.
\end{equation}
\req{NFFa1} is useful because while $\sgn^{2}(0)$ has an indeterminate value, $\sinh(0)=0$ vanishes; therefore we will not worry about this indeterminacy at the origin.

Using \req{NFF2a}, we replace the step function and exponential functions in \req{NFF1} with the following expressions
\begin{align}
\label{NFF4}
&\Theta(-x)=\frac 1 2 [1-\sgn(x)]\,,\\ 
&e^{-|x|}=\cosh|x|-\sinh|x|=\cosh(x)- \sgn(x)\sinh(x)\,.
\end{align}
Therefore we can rewrite the FD distribution \req{NFF1} as
\begin{align}
\begin{split}
\label{NFF4a}
f_\mathrm{FD}(x)&=\frac{1}{2}\left[1-\sinh(x)+\cosh(x)\tanh(x/2)\right]\\
&+\frac{1}{2}\sgn(x)\left[\cosh(x)-1-\sinh(x)\tanh(x/2)\right]\,.
\end{split}
\end{align}
Using the properties of the hyperbolic functions
\begin{align}
\cosh(x)-1=2\sinh^2(x/2)\,,\qquad
\sinh(x)=2\sinh(x/2)\cosh(x/2)\,,
\end{align}
\req{NFF4a} then simplifies to
\begin{align}
\begin{split}
\label{NFF4b}
f_\mathrm{FD}(x)
&=\frac{1}{2}\left[1-\sinh(x)+\cosh(x)\tanh(x/2)\right]\,,\\
&=\frac{1}{2}\left[1-\tanh(x/2)\right]\,,\\
&=\left(e^{x}+1\right)^{-1}\,,
\end{split}
\end{align}
which is just the original FD distribution written in the usual manner \req{f_old}.

\rev{This I believe is not necessary...if the functions match $\forall x \in \mathbb{R}$ then of course their derivatives match too.}Lastly, we check that the properties of the first derivative of \req{NFF1} are in agreement with the usual FD distribution. We write the first derivative of the singular distributions as 
\begin{align}
\label{NFF1b}
\frac{d}{dx}\Theta(-x)&=-\delta(x)\,,\qquad 
\frac{d}{dE}\Theta(-x)=-\frac{1}{T}\delta(x)\,,\\
\frac{d}{dx}\sgn(x)&=2\delta(x)\,,\,\,\quad 
\frac{d}{dE}\sgn(x)=\frac{2}{T}\delta(x)\,,
\end{align}
both without and with units and where $\delta(x)$ is the Dirac $\delta$-function. These cancel in \req{NFF1} at $x=0$ exactly as required since the derivative of the FD distribution written in \req{f_old} lacks a $\delta$-function. This encourages us to believe that all of singular expressions cancel leaving it fully analytic. This completes our demonstration of the validity of \req{NFF1}.

%%%%%%%%%%%%%%%%%%%%%%%%%%%%%%%%%%%%%%%%%%%%%%%%%%%%%%%%%%%%%%%%%%%%%%%
%Magnetization Example for future project
%%%%%%%%%%%%%%%%%%%%%%%%%%%%%%%%%%%%%%%%%%%%%%%%%%%%%%%%%%%%%%%%%%%%%%%
\section{Magnetization Example}
\label{magnetization}
{\bf ??Move to separate document??}
\noindent As an application of the novel form of the Fermi distribution, we introduce the problem of relativistic charged/magnetic gasses in a nearly homogeneous and isotropic fields. This problem has applicability in the primordial universe when such gasses were dense and rapidly cooling.

The grand partition function for the relativistic Fermi-Dirac ensemble is given by the standard definition
\begin{alignat}{1}
    \label{part:1} \ln\mathcal{Z}_\mathrm{total}=\sum_{\alpha}\ln\left(1+\Upsilon_{\alpha_{1}\ldots\alpha_{m}}\exp\left(-\frac{E_{\alpha}}{T}\right)\right)\,,\qquad\Upsilon_{\alpha_{1}\ldots\alpha_{m}}=\lambda_{\alpha_{1}}\lambda_{\alpha_{2}}\ldots\lambda_{\alpha_{m}}
\end{alignat}
where we are summing over the set all relevant quantum numbers $\alpha=(\alpha_{1},\alpha_{2},\ldots,\alpha_{m})$. We note here the generalized the fugacity $\Upsilon_{\alpha_{1}\ldots\alpha_{m}}$ allowing for any possible deformation caused by pressures $\lambda_{\alpha_{i}}$ effecting the distribution of any quantum numbers.

In the case of the Landau problem where the gas is made up of charged particles such as electrons and positrons, there is an additional summation over $\widetilde{w}$ which represents the occupancy of Landau states~\citep{greiner2012thermodynamics} which are matched to the available phase space within $\Delta p_{x}\Delta p_{y}$. If we consider the orbital Landau quantum number $n$ to represent the transverse momentum $p_{T}^{2}=p_{x}^{2}+p_{y}^{2}$ of the system, then the relationship that defines $\widetilde{w}$ is given by
\begin{alignat}{1}
    \label{phase:1} \frac{L^{2}}{(2\pi)^{2}}\Delta p_{x}\Delta p_{y}=\frac{eBL^{2}}{2\pi}\Delta n\,,\qquad\widetilde{w}=\frac{eBL^{2}}{2\pi}\,.
\end{alignat}
The summation over the continuous $p_{z}$ is replaced with an integration and the double summation over $p_{x}$ and $p_{y}$ is replaced by a single sum over Landau orbits
\begin{alignat}{1}
    \label{phase:2}
    \sum_{p_{z}}\rightarrow\frac{L}{2\pi}\int^{+\infty}_{-\infty}dp_{z}\,,\qquad\sum_{p_{x}}\sum_{p_{y}}\rightarrow\frac{eBL^{2}}{2\pi}\sum_{n}\,,
\end{alignat}
where $L$ defines the boundary length of our considered volume $V=L^{3}$.

The partition function of the $e^{+}e^{-}$ plasma can be understood as the sum of four gaseous species
\begin{align}
    \label{partition:0}    
    \ln\mathcal{Z}_{e^{+}e^{-}}=\ln\mathcal{Z}_{e^{+}}^{\uparrow}+\ln\mathcal{Z}_{e^{+}}^{\downarrow}+\ln\mathcal{Z}_{e^{-}}^{\uparrow}+\ln\mathcal{Z}_{e^{-}}^{\downarrow}\,,
\end{align}
of electrons and positrons of both polarizations $(\uparrow\downarrow)$. The change in phase space written in \req{phase:2} modify the magnetized $e^{+}e^{-}$ plasma partition function from \req{part:1} into
\begin{gather}
     \label{partition:1}
     \ln\mathcal{Z}_{e^{+}e^{-}}=\frac{e{B}V}{(2\pi)^{2}}\sum_{\sigma}^{\pm1}\sum_{s}^{\pm1}\sum_{n=0}^{\infty}\int_{-\infty}^{\infty}\mathrm{d}p_{z}\left[\ln\left(1+\lambda_{\sigma}\exp\left(-\frac{E_{\sigma,s}^{n}}{T}\right)\right)\right]\,\\
    \label{partition:2}    
    \Upsilon_{\sigma,s} \rightarrow\lambda_{\sigma} = \exp{\frac{\mu_{\sigma}}{T}}\,,
\end{gather}
where the energy eigenvalues $E_{\sigma,s}^{n}$ are given by
\begin{align}
 \label{cosmokgp}
 E^{n}_{\sigma,s}(p_{z},{B})=\sqrt{m_{e}^{2}+p_{z}^{2}+e{B}\left(2n+1+\frac{g}{2}\sigma s\right)}\,,
\end{align}

% Notes to self: We need a density (chemical potential) value for neutron stars. This tells us the amount of positrons and neutrons.

% We want three quantities: The energy <E>, pressure <P>, and particle density <n>.

The index $\sigma$ in \req{partition:1} is a sum over electron and positron states while $s$ is a sum over polarizations. The index $s$ refers to the spin along the field axis: parallel $(\uparrow;\ s=+1)$ or anti-parallel $(\downarrow;\ s=-1)$ for both particle and antiparticle species.

We are explicitly interested in small asymmetries such as baryon excess over antibaryons, or one polarization over another. For matter $(e^{-};\ \sigma=+1)$ and antimatter $(e^{+};\ \sigma=-1)$ particles, a nonzero relativistic chemical potential $\mu_{\sigma}=\sigma\mu$ is caused by an imbalance of matter and antimatter. While the primordial electron-positron plasma era was overall charge neutral, there was a small asymmetry in the charged leptons (namely electrons) from baryon asymmetry~\citep{Fromerth:2012fe,Canetti:2012zc} in the universe. Reactions such as $e^{+}e^{-}\leftrightarrow\gamma\gamma$ constrains the chemical potential of electrons and positrons~\citep{Elze:1980er} as 
\begin{align}
 \label{cpotential}
 \mu\equiv\mu_{e^{-}}=-\mu_{e^{+}}\,,\qquad
 \lambda\equiv\lambda_{e^{-}}=\lambda_{e^{+}}^{-1}=\exp\frac{\mu}{T}\,,
\end{align}
where $\lambda$ is the chemical fugacity of the system.

We can then parameterize the chemical potential of the $e^{+}e^{-}$ plasma as a function of temperature $\mu\rightarrow\mu(T)$ via the charge neutrality of the universe which implies
\begin{align}
 \label{chargeneutrality}
 n_{p}=n_{e^{-}}-n_{e^{+}}=\frac{1}{V}\lambda\frac{\partial}{\partial\lambda}\ln\mathcal{Z}_{e^{+}e^{-}}\,.
\end{align}
In \req{chargeneutrality}, $n_{p}$ is the observed total number density of protons in all baryon species. The chemical potential defined in \req{cpotential} is obtained from the requirement that the positive charge of baryons (protons, $\alpha$ particles, light nuclei produced after BBN) is exactly and locally compensated by a tiny net excess of electrons over positrons.
