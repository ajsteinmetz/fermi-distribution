% OLD MATH PROOF SECTION
Considering the sign and step function properties given in \req{NFF2}, we see that \req{NFF1} is equal to $1/2$ when $x=0$. This is the expected behavior of the FD distribution when the energy is equal to the chemical potential. To further demonstrate that \req{NFF1} is indeed the FD distribution, we will use the following properties of the sign and step functions
\begin{align}
\label{NFF2a}
\sgn(x)\equiv\frac{|x|}{x}\equiv\frac{x}{|x|}\,,\quad
\sgn(x)=2\Theta(x)-1\,,\quad
\Theta(x)+\Theta(-x)=1\,.
\end{align}
We also write the following hyperbolic expression
\begin{equation}
\label{NFFa1}
\sgn^{2}(x)\sinh(x)=\sinh(x)\,.
\end{equation}
\req{NFFa1} is useful because while $\sgn^{2}(0)$ has an indeterminate value, $\sinh(0)=0$ vanishes; therefore we will not worry about this indeterminacy at the origin.

Using \req{NFF2a}, we replace the step function and exponential functions in \req{NFF1} with the following expressions
\begin{align}
\label{NFF4}
&\Theta(-x)=\frac 1 2 [1-\sgn(x)]\,,\\ 
&e^{-|x|}=\cosh|x|-\sinh|x|=\cosh(x)- \sgn(x)\sinh(x)\,.
\end{align}
Therefore we can rewrite the FD distribution \req{NFF1} as
\begin{align}
\begin{split}
\label{NFF4a}
f_\mathrm{FD}(x)&=\frac{1}{2}\left[1-\sinh(x)+\cosh(x)\tanh(x/2)\right]\\
&+\frac{1}{2}\sgn(x)\left[\cosh(x)-1-\sinh(x)\tanh(x/2)\right]\,.
\end{split}
\end{align}
Using the properties of the hyperbolic functions
\begin{align}
\cosh(x)-1=2\sinh^2(x/2)\,,\qquad
\sinh(x)=2\sinh(x/2)\cosh(x/2)\,,
\end{align}
\req{NFF4a} then simplifies to
\begin{align}
\begin{split}
\label{NFF4b}
f_\mathrm{FD}(x)
&=\frac{1}{2}\left[1-\sinh(x)+\cosh(x)\tanh(x/2)\right]\,,\\
&=\frac{1}{2}\left[1-\tanh(x/2)\right]\,,\\
&=\left(e^{x}+1\right)^{-1}\,,
\end{split}
\end{align}
which is just the original FD distribution written in the usual manner \req{f_old}.

\rev{This I believe is not necessary...if the functions match $\forall x \in \mathbb{R}$ then of course their derivatives match too.}Lastly, we check that the properties of the first derivative of \req{NFF1} are in agreement with the usual FD distribution. We write the first derivative of the singular distributions as 
\begin{align}
\label{NFF1b}
\frac{d}{dx}\Theta(-x)&=-\delta(x)\,,\qquad 
\frac{d}{dE}\Theta(-x)=-\frac{1}{T}\delta(x)\,,\\
\frac{d}{dx}\sgn(x)&=2\delta(x)\,,\,\,\quad 
\frac{d}{dE}\sgn(x)=\frac{2}{T}\delta(x)\,,
\end{align}
both without and with units and where $\delta(x)$ is the Dirac $\delta$-function. These cancel in \req{NFF1} at $x=0$ exactly as required since the derivative of the FD distribution written in \req{f_old} lacks a $\delta$-function. This encourages us to believe that all of singular expressions cancel leaving it fully analytic. This completes our demonstration of the validity of \req{NFF1}.