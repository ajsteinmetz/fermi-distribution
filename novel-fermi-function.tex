%%\documentclass[sn-nature]{sn-jnl}% Style for submissions to Nature Portfolio journals
%%\documentclass[sn-basic]{sn-jnl}% Basic Springer Nature Reference Style/Chemistry Reference Style
\documentclass[sn-mathphys,Numbered]{sn-jnl}
\usepackage{graphicx}%
\usepackage{multirow}%
\usepackage{amsmath,amssymb,amsfonts}%
\usepackage{amsthm}%
\usepackage{mathrsfs}%
\usepackage[title]{appendix}%
\usepackage{xcolor}%
\usepackage{textcomp}%
\usepackage{manyfoot}%
\usepackage{booktabs}%
\usepackage{algorithm}%
\usepackage{algorithmicx}%
\usepackage{algpseudocode}%
\usepackage{listings}%
\usepackage[normalem]{ulem}

% Make Orcid icon
\newcommand{\orcidicon}{\includegraphics[width=0.32cm]{orcid.pdf}}
\newcommand{\orc}[1]{\href{https://orcid.org/#1}{\orcidicon}}

% Author Orcid ID: Define per author
\newcommand{\orcA}{0000-0001-8217-1484}
\newcommand{\orcB}{0000-0001-5038-8427}
\newcommand{\orcC}{0000-0001-5474-2649}
\newcommand{\orcD}{0000-0003-2704-6474}

% List of useful macros
\newcommand{\req}[1]{Eq.~(\ref{#1})}
\newcommand{\rf}[1]{Fig.~{\ref{#1}}}
\newcommand{\rt}[1]{Table~{\ref{#1}}}
\newcommand{\rsec}[1]{Sec.~{\ref{#1}}}
\newcommand*{\TeV}{\text{ TeV}}
\newcommand*{\GeV}{\text{ GeV}}
\newcommand*{\MeV}{\text{ MeV}}
\newcommand*{\keV}{\text{ keV}}
\newcommand*{\eV}{\text{ eV}}
\newcommand*{\meV}{\text{ meV}}
\DeclareMathOperator{\sgn}{sgn}

% Useful macros for annotation
\newcommand*{\xred}{\color{red}}
\newcommand*{\xblue}{\color{black}}
\newcommand*{\xgreen}{\color{green}}
\newcommand{\rev}[1]{{\color{blue}#1}}
\newcommand*{\rcite}{{\xred (Citation?)}}

%\theoremstyle{thmstyleone}%
%\newtheorem{theorem}{Theorem}
%\newtheorem{proposition}[theorem]{Proposition}% 
%\theoremstyle{thmstyletwo}%
%\newtheorem{example}{Example}%
%\newtheorem{remark}{Remark}%
%\theoremstyle{thmstylethree}%
%\newtheorem{definition}{Definition}%

\raggedbottom
\begin{document}

%\title[Article Title]{Cold Ideal Fermi Gas}
\title[Article Title]{Decomposition of Fermi gas into zero and finite temperature components with examples}

%%=============================================================%%
%% Prefix	-> \pfx{Dr}
%% GivenName	-> \fnm{Joergen W.}
%% Particle	-> \spfx{van der} -> surname prefix
%% FamilyName	-> \sur{Ploeg}
%% Suffix	-> \sfx{IV}
%% NatureName	-> \tanm{Poet Laureate} -> Title after name
%% Degrees	-> \dgr{MSc, PhD}
%% \author*[1,2]{\pfx{Dr} \fnm{Joergen W.} \spfx{van der} \sur{Ploeg} \sfx{IV} \tanm{Poet Laureate} 
%%                 \dgr{MSc, PhD}}\email{iauthor@gmail.com}
%%=============================================================%%

\author[1]{\fnm{Cheng Tao} \sur{Yang\orc{\orcB}}}
\author[1,2]{\fnm{Martin} \sur{Formanek\orc{\orcD}}}
\author[1]{\fnm{Andrew} \sur{Steinmetz\orc{\orcC}}} 
\author[1]{\fnm{Johann} \sur{Rafelski\orc{\orcA}}}
%\email{iiiauthor@gmail.com}
%\equalcont{These authors contributed equally to this work.}

\affil[1]{\orgdiv{Department of Physics}, \orgname{The University of Arizona}, \city{Tucson}, \state{Arizona}, \postcode{85721}, \country{USA}}

\affil[2]{\orgdiv{ELI Beamlines Facility}, \orgname{The Extreme Light Infrastructure ERIC}, \orgaddress{ \postcode{252 41}, \city{Dolni Brezany}, \country{Czech Republic}}}

\abstract{The most interesting physics of finite temperature Fermi gasses occurs around the Fermi surface, which needs a mathematical tool that can capture the finite temperature behavior of the Fermi-Dirac distribution in an analytic fashion. We provide a novel form of the Fermi distribution that can separate the Fermi gas into zero and finite temperature components analytically which is useful for addressing physics beyond the zero temperature approximation.}


\date{To be published in International Journal of Theoretical Physics}
\keywords{Fermi distribution, Low temperature Fermi gas}

%%\pacs[JEL Classification]{D8, H51}
%%\pacs[MSC Classification]{35A01, 65L10, 65L12, 65L20, 65L70}

\maketitle

%%%%%%%%%%%%%%%%%%%%%%%%%%%%%%%%%%%%%%%
\section{Introduction}
\label{sec1}
%%%%%%%%%%%%%%%%%%%%%%%%%%%%%%%%%%%%%%%
Systems which obey the Fermi-Dirac (FD) distribution, such as finite temperature Fermi quantum gasses, are difficult to study outside the zero temperature limit because the numerical precision of finite temperature contribution rapidly vanishes\rcite. Numerical evaluations of the finite temperature behavior of systems also obscures the physics as the precise origin of thermodynamic features are shrouded; whereas analytic evaluations can show explicitly the mathematical origin of physical features. This is why the few analytic solutions that exist in physics are invaluable to study.

% Kapusta and Gale - many body theory
% Lebellac - text
% 

The most interesting physics of Fermi gasses occurs at finite temperatures~\cite{Elze:1980er} which necessitates a mathematical tool which captures the finite temperature behavior of the FD distribution in an analytic fashion. We provide a novel form of FD distribution that can separate the Fermi gas into zero and finite temperature components analytically which is useful addressing physics beyond the zero temperature approximation. We also tackle the case of the magnetized relativistic Fermi gas to demonstrate the usefulness of this new method. Other use cases would be for compact astrophysical systems (white dwarfs, neutron stars, quark stars)~\cite{Kaspi:2017fwg,Ferrer:2019xlr,Ferrer:2023pgq}, Early Universe phenomenology~\cite{Rafelski:2021aey,Rafelski:2023emw,Grayson:2023flr,Steinmetz:2023nsc}, quark-gluon plasmas (QGP)~\cite{Letessier:2002ony,Rafelski:2020ajx,Yang:2021bko}, and wherever finite temperature Fermi effects are important.

Before we introduce the novel form in \rsec{NewFermi}, we will briefly recall the standard picture of the FD distribution. In the zero temperature limit $T\to0$, the FD distribution reduces to a step function where a state $E_{i}$ is either filled or empty. For given chemical potential $\widetilde\mu(T)$, we have
\begin{align}
\label{f_old}
f_\mathrm{FD}(E_{i},\widetilde\mu(T),T)=\left[\exp\left(\frac{E_{i}-\widetilde\mu}{T}\right)+1\right]^{-1}\,,\quad
\lim_{T\to0}f_\mathrm{FD}=\left\{
\begin{array}{c}
1,\quad\mathrm{for}\quad{E_{i}}<\widetilde\mu\\
0,\quad\mathrm{for}\quad{E_{i}}>\widetilde\mu
\end{array}
\right.\,.
\end{align}
The energy of the last filled state is called the Fermi energy and is denoted by $E_F$. The Fermi energy is also the value of the chemical potential at zero temperature $T=0$, \emph{i.e.} $E_F\equiv\widetilde\mu(T = 0)$.

%%%%%%%%%%%%%%%%%%%%%%%%%%%%%%%%%%%%%%%
\begin{figure}[ht]
\centering
\includegraphics[width=0.9\textwidth]{./plot/Electron_distribution001}
\caption{The FD distribution is plotted (solid blue line) as a function of energy with parameters $T=0.012\MeV$ and $\widetilde\mu=0.461\MeV$. The zero temperature distribution is plotted as the dashed black line.}
\label{Electron_001}
\end{figure}
%%%%%%%%%%%%%%%%%%%%%%%%%%%%%%%%%%%%%%%

In \rf{Electron_001} we plot the FD distribution of a fermion as a function of energy with example parameters $T=0.012\MeV$ and $\widetilde\mu=0.461\MeV$. This shows the textbook behavior that for finite temperatures, there is a filling of some higher energy states above the chemical potential $\widetilde\mu$ at the expense of states below $\widetilde\mu$. As the temperature increases, the distribution becomes more broad as a wide range of states become thermally populated.

%%%%%%%%%%%%%%%%%%%%%%%%%%%%%%%%%%%%%%%
\section{Zero and finite temperature Fermi-Dirac distributions}
\label{NewFermi}
%%%%%%%%%%%%%%%%%%%%%%%%%%%%%%%%%%%%%%%
\subsection{A novel form of Fermi-Dirac distribution}
\label{Novel}
%%%%%%%%%%%%%%%%%%%%%%%%%%%%%%%%%%%%%%%
Our interest and motivations in studying the Fermi-Dirac distribution were to perform cosmological computations involving both high and low temperature physics. In doing so, we have identified the following novel way to write FD distribution as three terms which separates out the zero temperature portion from the finite temperature contributions as 
\begin{align}
\label{NFF1}
\begin{split}
f_\mathrm{FD}(x)
&=\left(e^{x}+1\right)^{-1}\\
&=\Theta(-x)+\frac{1}{2}e^{-|x|}\left[\sgn(x)+\tanh(x/2)\right]\,,\quad
x\equiv\frac{E-\widetilde\mu}{T}\,,
\end{split}
\end{align}
\begin{align}
\label{NFF2}
\Theta(x)=\left\{
\begin{array}{r}
1,\quad\mathrm{for}\quad{x}>0\\
1/2,\quad\mathrm{for}\quad{x}=0\\
0,\quad\mathrm{for}\quad{x}<0
\end{array}\right.\,,\qquad
\sgn(x)=\left\{
\begin{array}{r}
+1,\quad\mathrm{for}\quad{x}>0\\
0,\quad\mathrm{for}\quad{x}=0\\
-1,\quad\mathrm{for}\quad{x}<0\\
\end{array}\right.\,,
\end{align}
where $\Theta(x)$ is the Heaviside step function and $\sgn(x)$ is the sign function. The first term $\Theta(-x)$ in \req{NFF1} represents the zero temperature portion of the FD distribution while the finite temperature terms are weighted by a decaying exponential function. The immediate benefit of this form is that numerical evaluations will naturally center around the Fermi surface of the system with finite temperature contributions exponentially weighted.

We note that \req{NFF1} comprises of distributions rather than analytical functions~\cite{Arfken:2011abc}. On first sight, it is difficult to see that these cancel to create the analytical FD function seen in \req{f_old}. In the following section, we will show that both sides are truly equivalent.

%%%%%%%%%%%%%%%%%%%%%%%%%%%%%%%%%%%%%%%
\subsection{Mathematical proof}
\label{Proof}
%%%%%%%%%%%%%%%%%%%%%%%%%%%%%%%%%%%%%%%
\rev{ALTERNATE PROOF:
For $x>0$
\begin{equation}
    f_\mathrm{FD}(x) = 0 + \frac{1}{2}e^{-x}[1+\tanh(x/2)] = (e^x + 1)^{-1}
\end{equation}
as can be seen by substituting $\tanh(x/2)=\sinh(x/2)/\cosh(x/2)$ and using hyperbolic formulas $\sinh(x/2)=(e^{x/2}-e^{-x/2})/2$, $\cosh(x/2) = (e^{x/2}+e^{-x/2})/2$. Similarly, for $x<0$
\begin{equation}
    f_\mathrm{FD}(x) = 1 + \frac{1}{2}e^{x}[-1 + \tanh(x/2)] = (e^x + 1)^{-1}\,.
\end{equation}
At the origin $x=0$ it can be shown that limits from the left and right are equal
\begin{align}
    \lim_{x\rightarrow 0^+} f_\mathrm{FD}(x) &= 0 + \frac{1}{2}(1 + 0) = \frac{1}{2}\\
    \lim_{x\rightarrow 0^-} f_\mathrm{FD}(x) &= 1 + \frac{1}{2}(-1 + 0) = \frac{1}{2}
\end{align}
which means that the limit of our expression at origin exists and is equal to the value of the function $(e^x+1)^{-1}$ at $x = 0$. With the definitions of $\Theta(x)$ and $\sgn(x)$ from Eq. (\ref{NFF2}) the limits for $x \rightarrow 0^\pm$ are even equal to the value of our function at origin. Therefore for all real $x$ our expression matches the original FD distribution Eq. (\ref{f_old}).  
}


Considering the sign and step function properties given in \req{NFF2}, we see that \req{NFF1} is equal to $1/2$ when $x=0$. This is the expected behavior of the FD distribution when the energy is equal to the chemical potential. To further demonstrate that \req{NFF1} is indeed the FD distribution, we will use the following properties of the sign and step functions
\begin{align}
\label{NFF2a}
\sgn(x)\equiv\frac{|x|}{x}\equiv\frac{x}{|x|}\,,\quad
\sgn(x)=2\Theta(x)-1\,,\quad
\Theta(x)+\Theta(-x)=1\,.
\end{align}
We also write the following hyperbolic expression
\begin{equation}
\label{NFFa1}
\sgn^{2}(x)\sinh(x)=\sinh(x)\,.
\end{equation}
\req{NFFa1} is useful because while $\sgn^{2}(0)$ has an indeterminate value, $\sinh(0)=0$ vanishes; therefore we will not worry about this indeterminacy at the origin.

Using \req{NFF2a}, we replace the step function and exponential functions in \req{NFF1} with the following expressions
\begin{align}
\label{NFF4}
&\Theta(-x)=\frac 1 2 [1-\sgn(x)]\,,\\ 
&e^{-|x|}=\cosh|x|-\sinh|x|=\cosh(x)- \sgn(x)\sinh(x)\,.
\end{align}
Therefore we can rewrite the FD distribution \req{NFF1} as
\begin{align}
\begin{split}
\label{NFF4a}
f_\mathrm{FD}(x)&=\frac{1}{2}\left[1-\sinh(x)+\cosh(x)\tanh(x/2)\right]\\
&+\frac{1}{2}\sgn(x)\left[\cosh(x)-1-\sinh(x)\tanh(x/2)\right]\,.
\end{split}
\end{align}
Using the properties of the hyperbolic functions
\begin{align}
\cosh(x)-1=2\sinh^2(x/2)\,,\qquad
\sinh(x)=2\sinh(x/2)\cosh(x/2)\,,
\end{align}
\req{NFF4a} then simplifies to
\begin{align}
\begin{split}
\label{NFF4b}
f_\mathrm{FD}(x)
&=\frac{1}{2}\left[1-\sinh(x)+\cosh(x)\tanh(x/2)\right]\,,\\
&=\frac{1}{2}\left[1-\tanh(x/2)\right]\,,\\
&=\left(e^{x}+1\right)^{-1}\,,
\end{split}
\end{align}
which is just the original FD distribution written in the usual manner \req{f_old}.

\rev{This I believe is not necessary...if the functions match $\forall x \in \mathbb{R}$ then of course their derivatives match too.}Lastly, we check that the properties of the first derivative of \req{NFF1} are in agreement with the usual FD distribution. We write the first derivative of the singular distributions as 
\begin{align}
\label{NFF1b}
\frac{d}{dx}\Theta(-x)&=-\delta(x)\,,\qquad 
\frac{d}{dE}\Theta(-x)=-\frac{1}{T}\delta(x)\,,\\
\frac{d}{dx}\sgn(x)&=2\delta(x)\,,\,\,\quad 
\frac{d}{dE}\sgn(x)=\frac{2}{T}\delta(x)\,,
\end{align}
both without and with units and where $\delta(x)$ is the Dirac $\delta$-function. These cancel in \req{NFF1} at $x=0$ exactly as required since the derivative of the FD distribution written in \req{f_old} lacks a $\delta$-function. This encourages us to believe that all of singular expressions cancel leaving it fully analytic. This completes our demonstration of the validity of \req{NFF1}. 

%%%%%%%%%%%%%%%%%%%%%%%%%%%%%%%%%%%%%%%
\subsection{Decomposition of zero and finite temperature contribution}
\label{Numerical}
%%%%%%%%%%%%%%%%%%%%%%%%%%%%%%%%%%%%%%%
%\sout{In addition to our mathematical derivation, we also present a graphical check to show that the novel expression for the distribution is well behaved numerically. In \rf{Fermi_Checking} we plot the traditional expression for the Fermi-distribution [\req{f_old}] with solid lines and the novel form of Fermi-distribution [\req{NFF1}] with dashed lines as a function of energy for various temperatures and chemical potential values. This demonstrates that \req{f_old} and \req{NFF1} are equivalent to each other numerically.}\rev{I would remove this and Fig.2. ...of course they are identical, we showed that they are. Or I would at least reframe Fig. 2. as examples of Fermi distribution for some parameters, not sure if needed. This way it makes an impression that we are not sure. I would focus on the splitting in Fig. 3.}

%%%%%%%%%%%%%%%%%%%%%%%%%%%%%%%%%%%%%%%
%\begin{figure}[ht]
%\centering
%\includegraphics[width=0.5\textwidth]{./plot/Fermi_novel_001}\includegraphics[width=0.5\textwidth]{./plot/Fermi_novel_002}
%\includegraphics[width=0.5\textwidth]{./plot/Fermi_novel_003}\includegraphics[width=0.5\textwidth]{./plot/Fermi_novel_004}
%\caption{We plot the Fermi-distribution~\req{f_old} (solid lines) against the novel formulation~\req{NFF1} (dashed lines) as a function of energy. (Upper plots) We compare the two formulations under different chemical potentials: $\widetilde\mu=(0,\pm0.1,\pm0.4,\pm0.6)\MeV$ at temperature $T=0.012\MeV$. (Lower plots) We compare under different temperatures $T=(0.511,0.0511, 0.012)\MeV$ for chemical potential $\widetilde\mu=\pm0.461\MeV$.}
%\label{Fermi_Checking}
%\end{figure}
%%%%%%%%%%%%%%%%%%%%%%%%%%%%%%%%%%%%%%%

To illustrate the zero and finite temperature contribution to the Fermi distribution function, it is convenient to rewrite Eq.~(\ref{NFF1}) into the following form
\begin{align}\label{Eq_form}
&f=f_{\mathrm{T=0}}+f_\mathrm{T\neq0}+\tilde f_\mathrm{T\neq0}
\end{align}
where the function are defined as
\begin{align}
&f_{\mathrm{T=0}}=\Theta\left(\frac{\widetilde\mu - E}{T}\right),\\ &f_\mathrm{T\neq0}=\frac{1}{2}e^{ - |E-\widetilde\mu|/T }\mathrm{sgn}\left(\frac{E-\widetilde\mu}{T}\right),\quad\tilde f_\mathrm{T\neq0}=\frac{1}{2}e^{ - |E-\widetilde\mu|/T }\tanh\left(\frac{E-\widetilde\mu}{2T}\right)\\
&f_\mathrm{T\neq0}=\frac{1}{2}e^{ - |E-\widetilde\mu|/T }\left[\mathrm{sgn}\left(\frac{E-\widetilde\mu}{T}\right)+\tanh\left(\frac{E-\widetilde\mu}{2T}\right)\right]
\end{align}
In Fig.~(\ref{Fermi_Component}) we plot the exact Fermi distribution (black line $f$), the zero (purple lines, $f_{T=0}$) and finite temperature components of the Fermi distribution as a function of energy choosing in this example the chemical potential $\widetilde\mu=0.461\MeV$ at temperature $T=0.02\MeV$ and at $T=0.2\MeV$.

%%%%%%%%%%%%%%%%%%%%%%%%%%%%%%%%%%%%%%%
\begin{figure}[ht]
\centering
\includegraphics[width=0.5\textwidth]{./plot/FermiZeorFiniteTemperature}\includegraphics[width=0.5\textwidth]{./plot/FermiZeroFiniteTemperature002}
\caption{%\rev{Here I would add the full fermi distribution - to see how it is decomposed to zero temperature and finite temperature terms.}
The zero and finite temperature components of the decomposition here considered for Fermi distribution as a function of energy with chemical potential $\widetilde\mu=0.461\MeV$ at temperature $T=0.02\MeV$ and $T=0.2\MeV$. The black line represents the exact Fermi distribution. The purple line represents the zero temperature component $f_{\mathrm{T}=0}$, and blue and red lines represent the finite temperature components $f_\mathrm{T\neq0}$ and $\tilde f_\mathrm{T\neq0}$ respectively. }
\label{Fermi_Component}
\end{figure}
%%%%%%%%%%%%%%%%%%%%%%%%%%%%%%%%%%%%%%%

The Fig.~(\ref{Fermi_Component}) shows that \rev{the contributions of the two finite temperature components $f_\mathrm{T\neq0}$ and $\tilde f_\mathrm{T\neq0}$ exponentially decay with the distance from the Fermi surface $E=\tilde{\mu}$. Moreover, the splitting was chosen in such a way that the discontinuous distribution part is fully contained in $f_\mathrm{T\neq 0}$ and $\tilde{f}_\mathrm{T\neq 0}$ is a remaining smooth function of $T$. Hmmm...actually the second derivative of $\tilde{f}_\mathrm{T\neq 0}$ is discontinuous at origin, is there any other motiviation for this splitting?}  Both finite temperature contributions always have the same sign \rev{and lower the distribution for $E < \tilde{\mu}$ and increase it for $E > \tilde{\mu}$}.
%\vfill\eject


{\xred(Comments for the numerical motivation:) In contrast to the brute force approach for eliminating the $T=0$ limit from the low-temperature Fermi distribution, our analytic form of distribution offers a more numerically advantageous method for separating the finite temperature components. This is advantage arise from the behavior of: 1.) both finite temperature contributions $f_\mathrm{T\neq0}$ and $\tilde f_\mathrm{T\neq0}$ always have the same sign 2.) the exponential factor for finite temperature components provide the naturally exponentially suppresses for the large energy. In this scenario, it reduces the numerical noise when evaluate the numerical integral beyond the zero temperature approximation. Furthermore, our novel form for Fermi distribution also provide us the tool to analytically investigate the finite temperature contributions by expanding the function around the Fermi-energy surface, which is useful for addressing physics for the finite temperature approximation.}



%%%%%%%%%%%%%%%%%%%%%%%%%%%%%%%%%%%%%%%
\section{Numerical illustration: Magnetization}
\label{NumericalResult}
%%%%%%%%%%%%%%%%%%%%%%%%%%%%%%%%%%%%%%%
In previous section we demonstrate that the Fermi-distribution can be written into a novel form Eq.~(\ref{NFF1}). To illustrate the benefits of our innovative Fermi distribution function in practical applications, it is convenient to introduce the partition function and consider the physical quantity magnetization~\cite{Melrose:2008abc} as a example.

Considering the partition function of $e^\pm$ gas in a uniform magnetic field $B$ pointing along the $z$-axis, we have
\begin{align}
\ln\mathcal{Z}_{tot}=eBV\!\!\sum_{s=\pm1}\sum_{j=0}^\infty\int^\infty_{-\infty} \!\!dp_z\bigg[\ln\left(1+e^{-\beta(E_{j,s}-\mu_e)}\right)+\ln\left(1+e^{-\beta(E_{j,s}+\mu_e)}\right)\bigg],
\end{align}
where electric charge $e\equiv q_{e^{+}}=-q_{e^{-}}$, $\beta=1/T$, $\mu_e$ is the chemical potential of electron, and the electron(positron) energy $E_{j,s}$ can be written as
\begin{align}
E_{j,s}&=\sqrt{m^2_e+p^2_z+2eB\left(j+\frac{1}{2}+\frac{g}{4}s\right)},\qquad s=\pm1,\qquad j=0,1,2,\dots
\end{align}
To simplify the partition function, we do integration by part and the partition function of $e^\pm$ gas becomes
\begin{align}
\ln\mathcal{Z}_{tot}&=eBV\sum_{s=\pm1}\sum_{j=0}^\infty2\beta\int_0^\infty\!\!dp_z \frac{p_z^2}{E_{j,s}}\left[\frac{1}{e^{\beta(E_{j,s}-\mu_e})+1}+\frac{1}{e^{\beta(E_{j,s}+\mu_e})+1}\right]\\
&=eBV\sum_{s=\pm1}\sum_{j=0}^\infty2\beta\int_0^\infty\!\!dp_z \frac{p_z^2}{E_{j,s}}\bigg[f_{e^-}+f_{e^+}\bigg].
\end{align}
Considering the case $g=2$, then we have
\begin{align}
&E_{j,+}=\sqrt{{m}_e^2+p^2_z+2eB\left(j+1\right)}\longrightarrow E_{n}=\sqrt{{m}_e^2+p^2_z+2eBn},\quad n=1,2,3,\dots\\
&E_{j,-}=\sqrt{{m}_e^2+p^2_z+2eB\left(j\right)}\longrightarrow E_{n}=\sqrt{{m}_e^2+p^2_z+2eBn},\,\, n=0,1,2,3,\dots
\end{align}
where we change the index from $j$ to $n$. In this case, the partition function of $e^\pm$ plasma can be written as
\begin{align}
\ln\mathcal{Z}_{tot}&=V(2eB\beta) \bigg[ \sum_{n=1}^\infty\int_0^\infty\!\!dp_z \frac{p_z^2}{E_{n}}\bigg(f_{e^-}(E_n)+f_{e^+}(E_n)\bigg)+\sum_{n=0}^\infty\int_0^\infty\!\!dp_z \frac{p_z^2}{E_{n}}\bigg(f_{e^-}(E_n)+f_{e^+}(E_n)\bigg)\bigg]\notag\\
&=V(2eB\beta)\bigg(\ln\mathcal{I}_{0}+\ln\mathcal{I}_{n}\bigg)
\end{align}
where the partition function $\ln\mathcal{I}_{0}$ and $\ln\mathcal{I}_{n}$ are defined as
\begin{align}
&\ln\mathcal{I}_{0}=\int_0^\infty\!\!dp_z \frac{p_z^2}{E_0}\bigg(f_{e^-}(E_0)+f_{e^+}(E_0)\bigg),\qquad E_0=\sqrt{{m}_e^2+p^2_z} \\
&\ln\mathcal{I}_{n}=2 \sum_{n=1}^\infty\int_0^\infty\!\!dp_z \frac{p_z^2}{E_{n}}\bigg(f_{e^-}(E_n)+f_{e^+}(E_n)\bigg),\quad E_n=\sqrt{{m}_e^2+p^2_z+2eBn}
\end{align}

Giving the partition function of $e^\pm$ gas in the magnetic field, the magnetization can be obtained via the definition
\begin{align}
M&=\frac{1}{V\beta}\frac{\partial \ln \mathcal{Z}_{tot}}{\partial B}=2e\frac{\partial}{\partial B}\bigg[B\bigg(\ln\mathcal{I}_{0}+\ln\mathcal{I}_{n}\bigg)\bigg]=M_0+M_n,\\
&M_0=2e\ln\mathcal{I}_{0},\qquad
\label{M_landau}
M_n=2e\bigg(\ln\mathcal{I}_{n}+B\frac{\partial\ln\mathcal{I}_n}{\partial B}\bigg)
\end{align}
where the $M_0$ represents the background magnetization  and $M_n$ is magnetization associate to Landau level (see Appendix for simplification). It is convenient to introduce the dimensionless variables and critical magnetic field $B_c$ as follows:
\begin{align}
\xi=p_z/T,\qquad \eta=m_e/T,\qquad B_c=m^2_e/e
\end{align}
In this case, the magnetization $M_0$ and $M_n$ can be written as
\begin{align}
&M_0=2eT^2\int_0^\infty\!\!d\xi\, F_0(E_0/T),\qquad M_n=2eT^2\int_{0}^\infty\!\!d\xi\left(2\sum_{n=1}^\infty\,F_n(E_n/T)\right)
\end{align}
where the function $F_n$ is defined as
\begin{align}
&F_n=\frac{(\xi^2-\eta^2nB/B_c)}{\sqrt{\xi^2+\eta^2\left(1+2nB/B_c\right)}}\bigg(f_{e^-}(E_n/T)+f_{e^+}(E_n/T)\bigg)\\
&E_n/T=\sqrt{\xi^2+\eta^2\left(1+2nB/B_c\right)},\qquad n=0,1,2,\cdots
\end{align}

To illustrate the benefits of our novel Fermi distribution function in calculation of magnetization, it is convenient to use the Fermi-function Eq.~(\ref{Eq001_appendix}) and the function $F_n$ can be written as
\begin{align}
&F_n=F_n^{\mathrm{T=0}}+F_n^{\mathrm{T\neq0}}+\widetilde F_n^{\mathrm{T\neq0}},
\end{align}
and we have
\begin{align}
&F_n^{\mathrm{T=0}}=\frac{(\xi^2-\eta^2nB/B_c)}{\sqrt{\xi^2+\eta^2\left(1+2nB/B_c\right)}}\bigg(f^{\mathrm{T=0}}_{e^-}(E_n/T)+f^{\mathrm{T=0}}_{e^+}(E_n/T)\bigg)\\
&F_n^{\mathrm{T\neq0}}=\frac{(\xi^2-\eta^2nB/B_c)}{\sqrt{\xi^2+\eta^2\left(1+2nB/B_c\right)}}\bigg(f^{\mathrm{T\neq0}}_{e^-}(E_n/T)+f^{\mathrm{T\neq0}}_{e^+}(E_n/T)\bigg)\\
&\widetilde F_n^{\mathrm{T\neq0}}=\frac{(\xi^2-\eta^2nB/B_c)}{\sqrt{\xi^2+\eta^2\left(1+2nB/B_c\right)}}\bigg(\tilde f^{\mathrm{T\ne0}}_{e^-}(E_n/T)+\tilde f^{\mathrm{T\neq0}}_{e^+}(E_n/T)\bigg)
\end{align}
In Fig~\ref{F0_Checking} (left) we plot the background  $F_0$ as a function of $p_z/T$ with temperature $T=0.012$ MeV and chemical potential $\mu=2$ MeV. The dominant term is the $F_{n=0}^{\mathrm{T=0}}$ which correspond to the step function contribution when $T=0$. The functions $F_{n=0}^{\mathrm{T\neq0}}$ and $\widetilde F_{n=0}^{\mathrm{T\neq0}}$ provide the correction near the energy surface $\mu/T\approx166.6$ MeV. In Fig~\ref{F0_Checking} (right) we plot the sum over $F_n$ as a function of $p_z/T$ with condition $T=0.012$ MeV, $\mu=2$ MeV and magnetic field $B/B_c=1/6$. It shows that for the above condition, the sum over Landau level is already stable for $N=50$.

%%%%%%%%%%%%%%%%%%%%%%%%%%%%%%%%%%%%%%%
\begin{figure}[ht]
\centering
\includegraphics[width=0.5\textwidth]{./plot/NewFermi_Background}\includegraphics[width=0.5\textwidth]{./plot/NewFermi_SumChecking}
\caption{ The function $F_n$ as a function of $p_z/T$ with conditions $T=0.012$ MeV, $\mu=2$ MeV and magnetic field $B/B_c=1/6$ as an example.
(Left) The function $F_n$ as a function of $p_z/T$ for $n=0$ (purple line). For comparison, the black line represents $F_{n=0}^{\mathrm{T=0}}$, the blue line label $F_{n=0}^{\mathrm{T\neq0}}$ and red line is $\tilde F_{n=0}^{\mathrm{T\neq0}}$. (Right) The sum over $\sum_{n=1}^NF_n$ as a function of $p_z/T$ for different values of $N$. It shows that for the given conditions, the sum over Landau level is converge for $N=50$.}
\label{F0_Checking}
\end{figure}
%%%%%%%%%%%%%%%%%%%%%%%%%%%%%%%%%%%%%%%
%%%%%%%%%%%%%%%%%%%%%%%%%%%%%%%%%%%%%%%
%\begin{figure}[ht]
%\centering
%\includegraphics[width=0.9\textwidth]{./plot/NewFermi_SumChecking}
%\caption{The sum over  $\sum_{n=1}^NF_n$ as a function of $p_z/T$ for different values of $N$. We consider the case: $T=0.012$ MeV, $\mu=2$ MeV and magnetic field $B/B_c=1/6$.}
%\label{Fn_Checking}
%\end{figure}
%%%%%%%%%%%%%%%%%%%%%%%%%%%%%%%%%%%%%%%

%%%%%%%%%%%%%%%%%%%%%%%%%%%%%%%%%%%%%%%
\begin{figure}[ht]
\centering
\includegraphics[width=0.5\textwidth]{./plot/NewFermi_Magnetization_tot003}\includegraphics[width=0.5\textwidth]{./plot/Magnetization_ChemicalPotential}
\caption{(Left) The magnetization $M/2eT^2$ as a function of temperature with the conditions $\mu=2$ MeV and $B/B_c=1/6$. It shows that the novel form of Fermi distribution improves the numerical calculation of Landau level by summing over finite terms $N$ for given temperature to obtain the converge results. (Right) The magnetization as a function of temperature with different chemical values from $2\,\mathrm{MeV}>\mu>0.05\,\mathrm{MeV}$ with a fixed value of $N=300$ for the Landau level summation.}
\label{M_Checking001}
\end{figure}
%%%%%%%%%%%%%%%%%%%%%%%%%%%%%%%%%%%%%%%
%%%%%%%%%%%%%%%%%%%%%%%%%%%%%%%%%%%%%%%%
%\begin{figure}[ht]
%\centering
%\includegraphics[width=0.9\textwidth]{./plot/Magnetization_ChemicalPotential}
%\caption{We plot the magnetization $M/2eT^2$ as a function of temperature with the conditions $N=300$ for sum over Landau level and $B/B_c=1/6$.}
%\label{M_Checking002}
%\end{figure}
%%%%%%%%%%%%%%%%%%%%%%%%%%%%%%%%%%%%%%%


In Fig.~\ref{M_Checking001} (left) we plot the magnetization $M/2eT^2$ as a function of temperature with the conditions $\mu=2$ MeV and $B/B_c=1/6$. It shows that the novel form of Fermi distribution can improve the numerical calculation of Landau level by summing over finite terms only. In cases where higher temperature ranges are involved, instead of summing over a vast number of $N$ terms, we only need to incorporate a few additional terms in the Landau level summation to achieve stable magnetization results. This results suggests that under the given conditions, our analytic form of Fermi function converges rapidly, significantly improving computational efficiency. In Fig.~\ref{M_Checking001} (right) we extend our analysis by plotting the magnetization as a function of temperature with different chemical values from $2\,\mathrm{MeV}>\mu>0.05\,\mathrm{MeV}$. This examination shows the versatility of our approach in addressing extreme conditions for the chemical potential of the Fermi gas. It shows the broad applicability of our approach and performance across different limit of Fermi system.







%%%%%%%%%%%%%%%%%%%%%%%%%%%%%%%%%%%%%%%

%%%%%%%%%%%%%%%%%%%%%%%%%%%%%%%%%%%%%%%
\section{Results and Discussion}
\label{sec12}
In this work, we introduce a novel form of the Fermi distribution Eq.~(\ref{NFF1}) that separate the Fermi gas into zero and finite temperature components analytically. This is the first time in literature that has clean separation of the zero and finite temperature components. Unlike the traditional brutal approach to eliminating the $T=0$ limit from the low-temperature Fermi distribution, our novel form of the Fermi distribution for finite temperature component has several numerical advantages in computation. 

The Fermi distribution Eq.~(\ref{NFF1}) provides us the tool to study or examine the effects and characteristics of finite temperature components separately. The mathematical form of the finite temperature components is well-suited for numerical calculations and can be used to address the integrals common in statistical physics. This is because the finite temperature components are naturally exponentially suppressed while also preserving the sign of all corrections making the formulation easier to numerically integrate and reduce the numerical noise, see Fig.~\ref{Fermi_Component}. In addition, our approach also allows for analytical exploration of the finite temperature contributions by expanding the function around the Fermi-energy surface.

To illustrate the numerical benefits, we examine the magnetization of electron-positron gas in a uniform magnetic filed as an example. Fig.~\ref{M_Checking001} shows that our novel form of Fermi distribution improves convergence of the numerical calculation of Landau levels. Instead of summing over a large number of terms, our approach  requires only a few additional terms in the Landau level summation to obtain the stable magnetization. This results imply that our analytic form of Fermi function exhibits rapid convergence and therefore improves the computational efficiency significantly.



%%%%%%%%%%%%%%%%%%%%%%%%%%%%%%%%%%%%%%%


% Martin - Highlight the method can be used to address the integrals common in stastistical physics.
% Johann - First time we separate cleanly the hot and cold components. This therefore could be very useful for studying interacting systems where the mathematics becomes intractible quickly. The properties near to the Fermi surface are clearly isolated and most of the interactions occur near the Fermi surface.
% Cold matter cannot form pairs, but finite temperature systems can because of the deformation of the step function. Particle-antiparticle hope pairs behavior is isolated. May have applications to superconducting systems. Allowing for the exploration of interacting systems at the Fermi surface.
% Reference to superconducting neutron stars.
% Andrew - This allows us to analyze the finite temperature elements as a distinct model.
% Cheng Tao - This novel form is numerically friendly as the finite temperature components are naturally exponentially suppresses while also preserving the sign of all corrections making the formulation easier to numerically integrate.

%%%%%%%%%%%%%%%%%%%%%%%%%%%%%%%%%%%%%%%
\section{Appendix}
\label{Append}
%%%%%%%%%%%%%%%%%%%%%%%%%%%%%%%%%%%%%%%
\subsection{Simplification of magnetization $M_n$}
The magnetization $M_n$ corresponds to the landau level contribution can be written as
\begin{align}
M_n=2e\bigg(\ln\mathcal{I}_{n}+B\frac{\partial\ln\mathcal{I}_n}{\partial B}\bigg)
\end{align}
Considering the second term in magnetization $M_n$, it can be written as
\begin{align}
B\frac{\partial\ln\mathcal{I}_n}{\partial B}&=2B \sum_{n=1}^\infty\int_0^\infty\!\!dp_z \frac{\partial E_n}{\partial B}\frac{\partial}{\partial E_n}\left[\frac{p_z^2}{E_{n}}\bigg(f_{e^-}(E_n)+f_{e^+}(E_n)\bigg)\right]\notag\\
%&=2B \sum_{n=1}^\infty\int_0^\infty\!\!dp_z \frac{en}{E_n}\frac{p_z^2}{E_{n}}\left[-\frac{1}{E_{n}}\bigg(f_{e^-}(E_n)+f_{e^+}(E_n)\bigg)+\bigg(\frac{\partial f_{e^-}(E_n)}{\partial E_n}+\frac{\partial f_{e^+}(E_n)}{\partial E_n}\bigg)\right]\notag\\
&=2eB \sum_{n=1}^\infty\int_0^\infty\!\!dp_z \frac{n p_z^2}{E^2_{n}}\left[-\frac{1}{E_{n}}\bigg(f_{e^-}(E_n)+f_{e^+}(E_n)\bigg)+\bigg(\frac{\partial f_{e^-}(E_n)}{\partial E_n}+\frac{\partial f_{e^+}(E_n)}{\partial E_n}\bigg)\right]\notag\\
&=2eB\sum_{n=1}^\infty n\bigg(A_n-D_n\bigg)\label{Eq001_appendix}
\end{align}
where the function $A_n$ and $D_n$ are defined as
\begin{align}
\label{Function_A}
&A_n=\int_{m_n}^\infty\!\!dE_n \frac{\sqrt{E^2_n-m^2_n}}{E_{n}}\bigg(\frac{\partial f_{e^-}(E_n)}{\partial E_n}+\frac{\partial f_{e^+}(E_n)}{\partial E_n}\bigg),\qquad m_n^2=m^2_e+2eBn\\ 
\label{Function_D}
&D_n=\int_{m_n}^\infty\!\!dE_n \frac{\sqrt{E^2_n-m^2_n}}{E^2_{n}}\bigg(f_{e^-}(E_n)+f_{e^+}(E_n)\bigg)
\end{align}
where we change the integral variable to energy and introduce the variable $m_n^2=m^2_e+2eBn$. Integrating by part, the function $A_n$ becomes
\begin{align}
A_n&=-\int_{m_n}^\infty\!\!dE_n \bigg(f_{e^-}(E_n)+f_{e^+}(E_n)\bigg)\frac{\partial}{\partial E_n}\left[\frac{\sqrt{E^2_n-m^2_n}}{E_{n}}\right]\notag\\
&=-\int_{m_n}^\infty\!\!dE_n \bigg(f_{e^-}(E_n)+f_{e^+}(E_n)\bigg)\left[\frac{m^2_n}{E^2_{n}\sqrt{E^2_n-m^2_n}}\right].
\end{align}
Substituting the $A_n$ and $D_n$ into Eq.~(\ref{Eq001_appendix}), we can simplify the equation as below:
\begin{align}
B\frac{\partial\ln\mathcal{I}_n}{\partial B}
&=2eB\sum_{n=1}^\infty n\int_{m_n}^\infty\!\!dE_n \bigg(f_{e^-}(E_n)+f_{e^+}(E_n)\bigg)\left[-\frac{m^2_n}{E^2_{n}\sqrt{E^2_n-m^2_n}}-\frac{\sqrt{E^2_n-m^2_n}}{E^2_{n}}\right]\notag\\
&=-2eB\sum_{n=1}^\infty n\int_{m_n}^\infty\!\!\frac{dE_n}{\sqrt{E^2_n-m^2_n}} \bigg(f_{e^-}(E_n)+f_{e^+}(E_n)\bigg).
\end{align}
In this case the magnetization $M_n$ can be written as
\begin{align}
M_n&=4e\sum_{n=1}^\infty \int_{m_n}^\infty\!\!dE_n\bigg(f_{e^-}(E_n)+f_{e^+}(E_n)\bigg)\left[{\sqrt{E^2_n-m^2_n}}-\frac{Ben}{\sqrt{E^2_n-m^2_n}}\right]\\
%&=4e\sum_{n=1}^\infty \int_{0}^\infty\!\!dp_z\frac{p_z}{E_n}\bigg(f_{e^-}(E_n)+f_{e^+}(E_n)\bigg)\left[\frac{p_z^2-Ben}{p_z}\right]\\
%&=4e\sum_{n=1}^\infty \int_{0}^\infty\!\!dp_z\frac{(p_z^2-Ben)}{E_n}\bigg(f_{e^-}(E_n)+f_{e^+}(E_n)\bigg)\\
&=4e \int_{0}^\infty\!\!dp_z\sum_{n=1}^\infty\frac{(p_z^2-Ben)}{E_n}\bigg(f_{e^-}(E_n)+f_{e^+}(E_n)\bigg).
\end{align}

%%%%%%%%%%%%%%%%%%%%%%%%%%%%%%%%%%%%%%%
\subsection{Numerical illustration}
For any given function $g(E)$, then the correspond physical quantity can be written as
\begin{align}
G\equiv\int^{\infty}_{m}dE\,g(E)\,f(E)=\int^{\infty}_{m}dE\,g(E)\bigg[f_{\mathrm{T=0}}+f_\mathrm{T\neq0}+\tilde f_\mathrm{T\neq0}\bigg]
\end{align}
To illustrate the zero and finite temperature contribution to the Fermi distribution function, it is convenient to rewrite Eq.~(\ref{NFF1}) into the following form
\begin{align}\label{Eq_form}
&f=f_{\mathrm{T=0}}+f_\mathrm{T\neq0}+\tilde f_\mathrm{T\neq0}
\end{align}
where the function are defined as
\begin{align}
&f_{\mathrm{T=0}}=\Theta\left(\frac{\widetilde\mu - E}{T}\right),\\ &f_\mathrm{T\neq0}=\frac{1}{2}e^{ - |E-\widetilde\mu|/T }\mathrm{sgn}\left(\frac{E-\widetilde\mu}{T}\right),\quad\tilde f_\mathrm{T\neq0}=\frac{1}{2}e^{ - |E-\widetilde\mu|/T }\tanh\left(\frac{E-\widetilde\mu}{2T}\right)\\
&f_\mathrm{T\neq0}=\frac{1}{2}e^{ - |E-\widetilde\mu|/T }\left[\mathrm{sgn}\left(\frac{E-\widetilde\mu}{T}\right)+\tanh\left(\frac{E-\widetilde\mu}{2T}\right)\right]
\end{align}
Integrating by part, the quantity $A$ can be written as
\begin{align}
\label{N_electron}
A=\int^{\infty}_{m_e}dE\,\rho(E)\,f_e(E)=A_{cold}+\int^{\infty}_{m_e}\,dE\left(g(E)\,f_\delta(E)-g^\prime(E)\,f^{(2)}_\delta(E)\right)+\left.g(E)\,f^{(2)}_\delta\right|_{surface},
\end{align}
where $g^\prime(e)=dg(E)/dE$ and the function $f^{(2)}_\delta(E)$ is defined as
\begin{align}
f^{(2)}_\delta(E)=\int^E_{m_e}dE^\prime\,f_\epsilon(E^\prime).
\end{align}







%%%%%%%%%%%%%%%%%%%%%%%%%%%%%%%%%%%%%%%
\backmatter

\bmhead{Acknowledgments}
We thank Gordon Baym and John W. Clark for their encouragement to publish this result.

%%%%%%%%%%%%%%%%%%%%%%%%%%%%%%%%%%%%%%%
\bibliography{novel-fermi-function-refs}
%% if required, the content of .bbl file can be included here once bbl is generated
%%\input sn-article.bbl
\end{document}
